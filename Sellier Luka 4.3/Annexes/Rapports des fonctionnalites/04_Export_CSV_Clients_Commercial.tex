\documentclass[12pt,a4paper]{article}
\usepackage[utf8]{inputenc}
\usepackage[T1]{fontenc}
\usepackage[french]{babel}
\usepackage{lmodern}
\usepackage{graphicx}
\usepackage{listings}
\usepackage{xcolor}
\usepackage{hyperref}
\usepackage{enumitem}
\usepackage{geometry}

\geometry{margin=2.5cm}

\hypersetup{
    colorlinks=true,
    linkcolor=blue,
    filecolor=magenta,
    urlcolor=cyan,
}

\lstset{
    language=PHP,
    basicstyle=\ttfamily\small,
    keywordstyle=\color{blue},
    stringstyle=\color{red},
    commentstyle=\color{green!60!black},
    numbers=left,
    numberstyle=\tiny\color{gray},
    stepnumber=1,
    numbersep=5pt,
    backgroundcolor=\color{gray!10},
    showspaces=false,
    showstringspaces=false,
    showtabs=false,
    frame=single,
    tabsize=4,
    captionpos=b,
    breaklines=true,
    breakatwhitespace=false,
    title=\lstname,
    escapeinside={},
    keywordstyle={},
    morekeywords={}
}

\title{Rapport de fonctionnalité : Export CSV des clients du commercial}
\author{NaturaCorp}
\date{\today}

\begin{document}

\maketitle

\section{Description}
Cette fonctionnalité permet aux commerciaux d'exporter en format CSV la liste de leurs clients (pharmacies) acquis sur une période spécifique. L'export inclut toutes les informations pertinentes sur les pharmacies, y compris leurs coordonnées et leur zone géographique.

\section{Objectifs}
\begin{itemize}
    \item Permettre l'extraction des données clients pour analyse externe
    \item Fournir un format compatible avec les outils de tableur (Excel, LibreOffice Calc)
    \item Inclure toutes les informations pertinentes sur les pharmacies
    \item Filtrer les données par période d'acquisition
\end{itemize}

\section{Implémentation technique}

\subsection{Fichiers concernés}
\begin{itemize}
    \item \texttt{app/Http/Controllers/ExportController.php} : Gestion de l'export CSV
    \item \texttt{resources/views/dashboard.blade.php} : Boutons d'export et sélecteurs de période
\end{itemize}

\subsection{Méthode principale}
La méthode \texttt{exportCommercialClients} dans le \texttt{ExportController.php} est responsable de l'export des clients :

\begin{lstlisting}[caption=Méthode exportCommercialClients]
public function exportCommercialClients(Request $request)
{
    $user = auth()->user();
    
    // Vérifier que l'utilisateur est un commercial
    if ($user->role !== 'commercial') {
        return redirect()->back()->with('error', 'Accès non autorisé');
    }
    
    // Récupérer les dates de début et de fin
    $startDate = $request->input('start_date') 
        ? Carbon::parse($request->input('start_date')) 
        : Carbon::now()->subDays(30);
    $endDate = $request->input('end_date') 
        ? Carbon::parse($request->input('end_date')) 
        : Carbon::now();
    
    // Handle period_type if provided
    if ($request->has('period_type') && !$request->has('start_date')) {
        // Cette méthode définit les dates de début et de fin en fonction du type de période
        // Exemple: 'last_30_days', 'last_3_months', 'last_6_months', 'last_year'
        $this->setPeriodDates($request->input('period_type'), $startDate, $endDate);
    }

    // Requête principale pour récupérer les clients (pharmacies) du commercial
    // SELECT * FROM pharmacies 
    // WHERE commercial_id = ? 
    // AND created_at BETWEEN ? AND ? 
    // ORDER BY created_at DESC
    $clients = Pharmacy::where('commercial_id', $user->id)
        ->whereBetween('created_at', [$startDate, $endDate])
        // Chargement eager de la relation zone pour éviter le problème N+1
        // Cela ajoute automatiquement un LEFT JOIN avec la table zones
        ->with('zone')
        ->get();
    
    // Générer le nom du fichier
    $filename = 'clients_' . $user->first_name . '_' . $user->last_name . '_' . Carbon::now()->format('Ymd_His') . '.csv';
    
    // Utilisation de la méthode streamDownload pour optimiser la mémoire
    return response()->streamDownload(function() use ($clients) {
        // Ouvrir un flux de sortie PHP
        $handle = fopen('php://output', 'w');
        
        // Écrire l'en-tête UTF-8 BOM pour Excel
        fprintf($handle, chr(0xEF).chr(0xBB).chr(0xBF));
        
        // Écrire les en-têtes du CSV
        fputcsv($handle, [
            'Date création',
            'Nom pharmacie',
            'Adresse',
            'Code postal',
            'Ville',
            'Téléphone',
            'Email',
            'Zone'
        ]);
        
        // Écrire les données
        foreach ($clients as $client) {
            fputcsv($handle, [
                $client->created_at->format('d/m/Y'),
                $client->name,
                $client->address,
                $client->postal_code,
                $client->city,
                $client->phone,
                $client->email,
                $client->zone ? $client->zone->name : 'Non assigné'
            ]);
        }
        
        fclose($handle);
    }, $filename);
}
\end{lstlisting}

\subsection{Traitement des données}
\begin{itemize}
    \item Les clients sont récupérés à partir des pharmacies assignées au commercial
    \item Les données sont filtrées par la date de création et par l'ID du commercial
    \item La relation avec la zone est chargée de manière optimisée (eager loading)
    \item Les données sont formatées pour être compatibles avec Excel (encodage UTF-8 avec BOM)
\end{itemize}

\subsection{Optimisation de la mémoire}
\begin{itemize}
    \item Utilisation de \texttt{response()->streamDownload()} pour générer le fichier en streaming
    \item Évite de charger l'ensemble du fichier CSV en mémoire
    \item Permet de gérer des exports volumineux sans problème de mémoire
\end{itemize}

\section{Avantages}
\begin{itemize}
    \item Export complet des données clients pour analyse externe
    \item Format CSV compatible avec tous les outils de tableur
    \item Filtrage flexible par période d'acquisition
    \item Optimisation de la mémoire pour gérer de grands volumes de données
    \item Inclusion de toutes les informations pertinentes pour l'analyse de la clientèle
\end{itemize}

\end{document}
