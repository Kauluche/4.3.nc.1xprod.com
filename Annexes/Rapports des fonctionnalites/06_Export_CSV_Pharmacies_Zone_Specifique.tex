\documentclass[12pt,a4paper]{article}
\usepackage[utf8]{inputenc}
\usepackage[T1]{fontenc}
\usepackage[french]{babel}
\usepackage{lmodern}
\usepackage{graphicx}
\usepackage{listings}
\usepackage{xcolor}
\usepackage{hyperref}
\usepackage{enumitem}
\usepackage{geometry}

\geometry{margin=2.5cm}

\hypersetup{
    colorlinks=true,
    linkcolor=blue,
    filecolor=magenta,
    urlcolor=cyan,
}

\lstset{
    language=PHP,
    basicstyle=\ttfamily\small,
    keywordstyle=\color{blue},
    stringstyle=\color{red},
    commentstyle=\color{green!60!black},
    numbers=left,
    numberstyle=\tiny\color{gray},
    stepnumber=1,
    numbersep=5pt,
    backgroundcolor=\color{gray!10},
    showspaces=false,
    showstringspaces=false,
    showtabs=false,
    frame=single,
    tabsize=4,
    captionpos=b,
    breaklines=true,
    breakatwhitespace=false,
    title=\lstname,
    escapeinside={},
    keywordstyle={},
    morekeywords={}
}

\title{Rapport de fonctionnalité : Export CSV des pharmacies d'une zone spécifique}
\author{NaturaCorp}
\date{\today}

\begin{document}

\maketitle

\section{Description}
Cette fonctionnalité permet aux administrateurs d'exporter en format CSV la liste des pharmacies appartenant à une zone géographique spécifique. L'export inclut toutes les informations pertinentes sur les pharmacies, y compris leurs coordonnées, leur commercial assigné et leurs statistiques de commandes.

\section{Objectifs}
\begin{itemize}
    \item Permettre l'extraction ciblée des données des pharmacies d'une zone spécifique
    \item Fournir un format compatible avec les outils de tableur (Excel, LibreOffice Calc)
    \item Inclure toutes les informations pertinentes sur les pharmacies
    \item Faciliter l'analyse détaillée d'une zone géographique particulière
\end{itemize}

\section{Implémentation technique}

\subsection{Fichiers concernés}
\begin{itemize}
    \item \texttt{app/Http/Controllers/ExportController.php} : Gestion de l'export CSV
    \item \texttt{resources/views/admin/reports/index.blade.php} : Boutons d'export dans l'interface administrateur
\end{itemize}

\subsection{Méthode principale}
La méthode \texttt{exportPharmaciesByZone} dans le \texttt{ExportController.php} est responsable de l'export des pharmacies d'une zone spécifique :

\begin{lstlisting}[caption=Méthode exportPharmaciesByZone]
public function exportPharmaciesByZone($zoneId)
{
    // Récupération d'une zone spécifique avec ses pharmacies associées
    // SELECT zones.*, pharmacies.* FROM zones
    // LEFT JOIN pharmacies ON zones.id = pharmacies.zone_id
    // WHERE zones.id = ?
    // LIMIT 1
    $zone = Zone::with('pharmacies')->findOrFail($zoneId);
    
    $filename = 'pharmacies_zone_' . $zone->name . '_' . Carbon::now()->format('Ymd_His') . '.csv';
    
    // Utilisation de la méthode streamDownload pour optimiser la mémoire
    return response()->streamDownload(function() use ($zone) {
        // Ouvrir un flux de sortie PHP
        $handle = fopen('php://output', 'w');
        
        // Écrire l'en-tête UTF-8 BOM pour Excel
        fprintf($handle, chr(0xEF).chr(0xBB).chr(0xBF));
        
        // Écrire les en-têtes du CSV
        fputcsv($handle, [
            'Nom pharmacie',
            'Adresse',
            'Code postal',
            'Ville',
            'Téléphone',
            'Email',
            'Commercial assigné',
            'Nombre de commandes',
            'Montant total des commandes',
            'Date dernière commande'
        ]);
        
        // Écrire les données pour chaque pharmacie de la zone
        foreach ($zone->pharmacies as $pharmacy) {
            // Calculer le nombre de commandes et le montant total
            $orders = Order::where('pharmacy_id', $pharmacy->id)->get();
            $orderCount = $orders->count();
            
            $totalAmount = Order::where('pharmacy_id', $pharmacy->id)
                ->join('order_items', 'orders.id', '=', 'order_items.order_id')
                ->sum(DB::raw('order_items.quantity * order_items.unit_price * (1 - order_items.discount_percentage / 100)'));
            
            // Récupérer la date de la dernière commande
            $lastOrderDate = $orders->max('created_at') 
                ? Carbon::parse($orders->max('created_at'))->format('d/m/Y') 
                : 'Aucune commande';
            
            // Récupérer le nom du commercial assigné
            $commercialName = $pharmacy->commercial 
                ? $pharmacy->commercial->first_name . ' ' . $pharmacy->commercial->last_name 
                : 'Non assigné';
            
            fputcsv($handle, [
                $pharmacy->name,
                $pharmacy->address,
                $pharmacy->postal_code,
                $pharmacy->city,
                $pharmacy->phone,
                $pharmacy->email,
                $commercialName,
                $orderCount,
                number_format($totalAmount, 2, ',', ' ') . ' €',
                $lastOrderDate
            ]);
        }
        
        fclose($handle);
    }, $filename);
}
\end{lstlisting}

\subsection{Traitement des données}
\begin{itemize}
    \item La zone spécifique et ses pharmacies associées sont récupérées avec un chargement eager
    \item Pour chaque pharmacie, le nombre de commandes, le montant total et la date de dernière commande sont calculés
    \item Le nom du commercial assigné est récupéré à partir de la relation
    \item Les données sont formatées pour être compatibles avec Excel (encodage UTF-8 avec BOM)
\end{itemize}

\subsection{Optimisation de la mémoire}
\begin{itemize}
    \item Utilisation de \texttt{response()->streamDownload()} pour générer le fichier en streaming
    \item Évite de charger l'ensemble du fichier CSV en mémoire
    \item Permet de gérer des exports volumineux sans problème de mémoire
\end{itemize}

\section{Avantages}
\begin{itemize}
    \item Export ciblé des données pharmacies d'une zone spécifique
    \item Format CSV compatible avec tous les outils de tableur
    \item Inclusion des statistiques de commandes pour évaluer la performance
    \item Optimisation de la mémoire pour gérer de grands volumes de données
    \item Facilite l'analyse détaillée d'une zone géographique particulière
\end{itemize}

\end{document}
