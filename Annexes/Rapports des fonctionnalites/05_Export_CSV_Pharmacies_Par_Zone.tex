\documentclass[12pt,a4paper]{article}
\usepackage[utf8]{inputenc}
\usepackage[T1]{fontenc}
\usepackage[french]{babel}
\usepackage{lmodern}
\usepackage{graphicx}
\usepackage{listings}
\usepackage{xcolor}
\usepackage{hyperref}
\usepackage{enumitem}
\usepackage{geometry}

\geometry{margin=2.5cm}

\hypersetup{
    colorlinks=true,
    linkcolor=blue,
    filecolor=magenta,
    urlcolor=cyan,
}

\lstset{
    language=PHP,
    basicstyle=\ttfamily\small,
    keywordstyle=\color{blue},
    stringstyle=\color{red},
    commentstyle=\color{green!60!black},
    numbers=left,
    numberstyle=\tiny\color{gray},
    stepnumber=1,
    numbersep=5pt,
    backgroundcolor=\color{gray!10},
    showspaces=false,
    showstringspaces=false,
    showtabs=false,
    frame=single,
    tabsize=4,
    captionpos=b,
    breaklines=true,
    breakatwhitespace=false,
    title=\lstname,
    escapeinside={},
    keywordstyle={},
    morekeywords={}
}

\title{Rapport de fonctionnalité : Export CSV des pharmacies par zone}
\author{NaturaCorp}
\date{\today}

\begin{document}

\maketitle

\section{Description}
Cette fonctionnalité permet aux administrateurs d'exporter en format CSV la liste complète des pharmacies regroupées par zone géographique. L'export inclut toutes les informations pertinentes sur les pharmacies, y compris leurs coordonnées, leur commercial assigné et leurs statistiques de commandes.

\section{Objectifs}
\begin{itemize}
    \item Permettre l'extraction des données de toutes les pharmacies pour analyse externe
    \item Fournir un format compatible avec les outils de tableur (Excel, LibreOffice Calc)
    \item Inclure toutes les informations pertinentes sur les pharmacies et leur zone
    \item Faciliter l'analyse de la répartition géographique des pharmacies
\end{itemize}

\section{Implémentation technique}

\subsection{Fichiers concernés}
\begin{itemize}
    \item \texttt{app/Http/Controllers/ExportController.php} : Gestion de l'export CSV
    \item \texttt{resources/views/admin/reports/index.blade.php} : Boutons d'export dans l'interface administrateur
\end{itemize}

\subsection{Méthode principale}
La méthode \texttt{exportAllPharmaciesByZone} dans le \texttt{ExportController.php} est responsable de l'export des pharmacies par zone :

\begin{lstlisting}[caption=Méthode exportAllPharmaciesByZone]
public function exportAllPharmaciesByZone()
{
    // Récupération de toutes les zones avec leurs pharmacies associées
    // SELECT zones.*, pharmacies.* FROM zones
    // LEFT JOIN pharmacies ON zones.id = pharmacies.zone_id
    // ORDER BY zones.name ASC
    $zones = Zone::with('pharmacies')->get();
    
    $filename = 'pharmacies_par_zone_' . Carbon::now()->format('Ymd_His') . '.csv';
    
    // Utilisation de la méthode streamDownload pour optimiser la mémoire
    return response()->streamDownload(function() use ($zones) {
        // Ouvrir un flux de sortie PHP
        $handle = fopen('php://output', 'w');
        
        // Écrire l'en-tête UTF-8 BOM pour Excel
        fprintf($handle, chr(0xEF).chr(0xBB).chr(0xBF));
        
        // Écrire les en-têtes du CSV
        fputcsv($handle, [
            'Zone',
            'Nom pharmacie',
            'Adresse',
            'Code postal',
            'Ville',
            'Téléphone',
            'Email',
            'Commercial assigné',
            'Nombre de commandes',
            'Montant total des commandes'
        ]);
        
        // Écrire les données, regroupées par zone
        foreach ($zones as $zone) {
            foreach ($zone->pharmacies as $pharmacy) {
                // Calculer le nombre de commandes et le montant total
                $orderCount = Order::where('pharmacy_id', $pharmacy->id)->count();
                
                $totalAmount = Order::where('pharmacy_id', $pharmacy->id)
                    ->join('order_items', 'orders.id', '=', 'order_items.order_id')
                    ->sum(DB::raw('order_items.quantity * order_items.unit_price * (1 - order_items.discount_percentage / 100)'));
                
                // Récupérer le nom du commercial assigné
                $commercialName = $pharmacy->commercial 
                    ? $pharmacy->commercial->first_name . ' ' . $pharmacy->commercial->last_name 
                    : 'Non assigné';
                
                fputcsv($handle, [
                    $zone->name,
                    $pharmacy->name,
                    $pharmacy->address,
                    $pharmacy->postal_code,
                    $pharmacy->city,
                    $pharmacy->phone,
                    $pharmacy->email,
                    $commercialName,
                    $orderCount,
                    number_format($totalAmount, 2, ',', ' ') . ' €'
                ]);
            }
        }
        
        fclose($handle);
    }, $filename);
}
\end{lstlisting}

\subsection{Traitement des données}
\begin{itemize}
    \item Les zones et leurs pharmacies associées sont récupérées avec un chargement eager
    \item Pour chaque pharmacie, le nombre de commandes et le montant total sont calculés
    \item Le nom du commercial assigné est récupéré à partir de la relation
    \item Les données sont formatées pour être compatibles avec Excel (encodage UTF-8 avec BOM)
\end{itemize}

\subsection{Optimisation de la mémoire}
\begin{itemize}
    \item Utilisation de \texttt{response()->streamDownload()} pour générer le fichier en streaming
    \item Évite de charger l'ensemble du fichier CSV en mémoire
    \item Permet de gérer des exports volumineux sans problème de mémoire
\end{itemize}

\section{Avantages}
\begin{itemize}
    \item Export complet des données pharmacies pour analyse externe
    \item Regroupement par zone géographique pour une analyse territoriale
    \item Format CSV compatible avec tous les outils de tableur
    \item Inclusion des statistiques de commandes pour évaluer la performance par zone
    \item Optimisation de la mémoire pour gérer de grands volumes de données
\end{itemize}

\end{document}
