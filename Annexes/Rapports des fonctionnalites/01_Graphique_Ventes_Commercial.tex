\documentclass[12pt,a4paper]{article}
\usepackage[utf8]{inputenc}
\usepackage[T1]{fontenc}
\usepackage[french]{babel}
\usepackage{lmodern}
\usepackage{graphicx}
\usepackage{listings}
\usepackage{xcolor}
\usepackage{hyperref}
\usepackage{enumitem}
\usepackage{geometry}

\geometry{margin=2.5cm}

\hypersetup{
    colorlinks=true,
    linkcolor=blue,
    filecolor=magenta,
    urlcolor=cyan,
}

\lstset{
    language=PHP,
    basicstyle=\ttfamily\small,
    keywordstyle=\color{blue},
    stringstyle=\color{red},
    commentstyle=\color{green!60!black},
    numbers=left,
    numberstyle=\tiny\color{gray},
    stepnumber=1,
    numbersep=5pt,
    backgroundcolor=\color{gray!10},
    showspaces=false,
    showstringspaces=false,
    showtabs=false,
    frame=single,
    tabsize=4,
    captionpos=b,
    breaklines=true,
    breakatwhitespace=false,
    title=\lstname,
    escapeinside={},
    keywordstyle={},
    morekeywords={}
}

\title{Rapport de fonctionnalité : Graphique des ventes du commercial}
\author{NaturaCorp}
\date{\today}

\begin{document}

\maketitle

\section{Description}
Cette fonctionnalité permet aux commerciaux de visualiser sous forme de graphique l'évolution de leurs ventes sur une période donnée. Le graphique présente les montants totaux des ventes agrégés par jour, semaine ou mois selon la période sélectionnée.

\section{Objectifs}
\begin{itemize}
    \item Offrir une visualisation claire de l'évolution des ventes dans le temps
    \item Permettre le filtrage par différentes périodes (30 derniers jours, 3 derniers mois, 6 derniers mois, dernière année)
    \item Adapter automatiquement la granularité des données selon la période sélectionnée
    \item Fournir une interface interactive pour consulter les détails au survol
\end{itemize}

\section{Implémentation technique}

\subsection{Fichiers concernés}
\begin{itemize}
    \item \texttt{app/Http/Controllers/DashboardController.php} : Préparation des données pour le graphique
    \item \texttt{resources/views/dashboard.blade.php} : Affichage du graphique et des contrôles de filtrage
    \item Bibliothèque externe : Chart.js pour le rendu du graphique
\end{itemize}

\subsection{Méthode principale}
La méthode \texttt{prepareSalesChartData} dans le \texttt{DashboardController.php} est responsable de la génération des données :

\begin{lstlisting}[caption=Méthode prepareSalesChartData]
private function prepareSalesChartData($user, Carbon $startDate, Carbon $endDate)
{
    $pharmacyIds = $user->pharmacies()->pluck('id')->toArray();
    
    $salesData = [];
    $labels = [];
    $interval = 'month'; // Par défaut, intervalle mensuel
    
    // Déterminer l'intervalle approprié en fonction de la durée
    $diffInDays = $startDate->diffInDays($endDate);
    
    if ($diffInDays <= 31) {
        // Période courte (moins d'un mois) : afficher par jour
        $interval = 'day';
    } else if ($diffInDays <= 90) {
        // Période moyenne (1-3 mois) : afficher par semaine
        $interval = 'week';
    }
    
    // Génération des données selon l'intervalle approprié
    $currentDate = clone $startDate;
    
    while ($currentDate <= $endDate) {
        // Déterminer la date de fin pour cette période selon l'intervalle
        if ($interval == 'day') {
            $nextDate = (clone $currentDate)->addDay();
            $label = $currentDate->format('d/m');
            
            // Requête SQL pour les ventes journalières
            $query = Order::whereIn('pharmacy_id', $pharmacyIds)
                ->whereDate('created_at', '=', $currentDate->format('Y-m-d'));
                
        } else if ($interval == 'week') {
            $nextDate = (clone $currentDate)->addWeek();
            $label = $currentDate->format('d/m') . ' - ' . (clone $nextDate)->subDay()->format('d/m');
            
            // Requête SQL pour les ventes hebdomadaires
            $query = Order::whereIn('pharmacy_id', $pharmacyIds)
                ->whereDate('created_at', '>=', $currentDate->format('Y-m-d'))
                ->whereDate('created_at', '<', $nextDate->format('Y-m-d'));
                
        } else {
            $nextDate = (clone $currentDate)->addMonth();
            $label = $currentDate->format('M Y');
            
            // Requête SQL pour les ventes mensuelles
            $query = Order::whereIn('pharmacy_id', $pharmacyIds)
                ->whereYear('created_at', $currentDate->year)
                ->whereMonth('created_at', $currentDate->month);
        }
        
        // Calcul du montant total des ventes pour cette période
        // Cette requête joint la table orders avec order_items pour calculer le montant réel
        // en tenant compte des quantités, prix unitaires et remises
        $periodSales = $query->join('order_items', 'orders.id', '=', 'order_items.order_id')
            ->select(DB::raw('SUM(order_items.quantity * order_items.unit_price * (1 - order_items.discount_percentage / 100)) as total_amount'))
            ->first()->total_amount ?? 0;
        
        // Ajout des données au tableau pour le graphique
        $labels[] = $label;
        $salesData[] = round($periodSales, 2);
        
        $currentDate = $nextDate;
    }
    
    return [
        'labels' => $labels,
        'data' => $salesData
    ];
}
\end{lstlisting}

\subsection{Calcul des données}
\begin{itemize}
    \item Les ventes sont calculées à partir des commandes liées aux pharmacies du commercial
    \item Le montant total des ventes est obtenu en multipliant la quantité par le prix unitaire et en appliquant les remises
    \item Les données sont agrégées par jour, semaine ou mois selon la durée de la période sélectionnée
\end{itemize}

\subsection{Interface utilisateur}
\begin{itemize}
    \item Graphique linéaire réalisé avec Chart.js
    \item Sélecteur de période (30 derniers jours, 3 derniers mois, 6 derniers mois, dernière année)
    \item Affichage des montants exacts au survol des points du graphique
    \item Mise à jour dynamique du graphique lors du changement de période
\end{itemize}

\section{Avantages}
\begin{itemize}
    \item Visualisation claire de l'évolution des performances commerciales
    \item Granularité adaptative selon la période pour une meilleure lisibilité
    \item Interface intuitive et réactive pour l'utilisateur
    \item Aide à l'identification des tendances et des périodes de forte/faible activité
\end{itemize}

\end{document}
