% Plan du rapport individuel NaturaCorp (version avancée)

% ===================
% Préambule avancé
% ===================

\documentclass[a4paper,12pt]{report}

% Numérotation personnalisée
\renewcommand{\thesection}{\arabic{section}}
\renewcommand{\thesubsection}{\thesection.\arabic{subsection}}

% Packages principaux
\usepackage[french]{babel}
\usepackage[utf8]{inputenc}
\usepackage[T1]{fontenc}
\usepackage{graphicx}
\usepackage{xcolor}
\usepackage{geometry}
\usepackage{titlesec}
\usepackage{array}
\usepackage{longtable}
\usepackage{booktabs}
\usepackage{caption}
\usepackage{hyperref}
\usepackage[most]{tcolorbox}
\usepackage{tikz}
\usepackage{listings}
\usepackage{enumitem}

% Configuration des liens hypertextes
\definecolor{naturacorpgreen}{RGB}{0,128,64}
\hypersetup{
    colorlinks=true,
    linkcolor=naturacorpgreen,
    urlcolor=blue,
    citecolor=blue,
    pdftitle={Rapport Individuel - NaturaCorp},
    pdfauthor={À compléter},
    pdfsubject={Documentation de maintenance évolutive},
    pdfkeywords={NaturaCorp, Maintenance, Évolution, Rapport},
    pdfstartview={FitH},
    bookmarksnumbered=true,
    pdfpagemode=UseOutlines
}

% Mise en page
\geometry{margin=2.5cm}

% Couleurs personnalisées
\definecolor{lightgray}{HTML}{F2F2F2}
\definecolor{codegreen}{rgb}{0,0.6,0}
\definecolor{codegray}{rgb}{0.5,0.5,0.5}
\definecolor{codepurple}{rgb}{0.58,0,0.82}
\definecolor{backcolour}{rgb}{0.95,0.95,0.92}

% Personnalisation des titres
\titleformat{\section}
  {\normalfont\Large\bfseries\color{naturacorpgreen}}
  {\thesection}{1em}{}
\titleformat{\subsection}
  {\normalfont\large\bfseries}
  {\thesubsection}{1em}{}

% En-tête et pied de page
\usepackage{fancyhdr}
\fancyhf{}
\fancyfoot[L]{Juin 2025}
\fancyfoot[C]{NaturaCorp - Maintenance évolutive}
\fancyfoot[R]{\thepage}
\renewcommand{\footrulewidth}{0.4pt}
\pagestyle{fancy}

% ===================
% Début du document
% ===================
\begin{document}

% --- PAGE DE GARDE ---
\thispagestyle{empty}
\begin{center}
    \begin{minipage}{0.45\textwidth}
        \centering
        \includegraphics[width=5cm]{naturacorp.png}
        \vspace{0.3cm}
    \end{minipage}
    \hfill
    \begin{minipage}{0.45\textwidth}
        \centering
        \includegraphics[width=5cm]{esn.jpeg}
        \vspace{0.3cm}
    \end{minipage}
    \vspace*{0.5cm}
    
    % Titre principal
    {\Huge\bfseries\color{naturacorpgreen} Rapport Individuel – Projet NaturaCorp\par}
    \vspace{1.2cm}
    
    % Sous-titre
    {\LARGE\bfseries Livrable 4.3 – Maintenance évolutive\par}
    \vspace{2cm}
    
    % Auteur
    {\Large\bfseries Réalisé par :\par}
    \vspace{0.3cm}
    {\Large SELLIER Luka\par}
    \vspace{0.5cm}
    {\large Formation : Bachelor 3 Développement WEB\par}
    \vspace{0.3cm}
    \vspace{0.3cm}
    {\large Année universitaire : 2024--2025\par}
    \vspace{1.5cm}
    
    % Informations projet
    \begin{minipage}{0.8\textwidth}
        \centering
        \textbf{Projet d’évolution fonctionnelle}\
        \vspace{0.2cm}
        Maintenance évolutive de la solution numérique NaturaCorp
    \end{minipage}
    \vspace{1.5cm}
    
    % Date
    {\large Juin 2025\par}
\vspace*{\fill}
\begin{center}
    \textbf{Livrable 4.3 : Maintenance évolutive}
\end{center}
\end{center}

\newpage
% --- SOMMAIRE ---
\renewcommand{\contentsname}{Sommaire}
\tableofcontents
\newpage

\section{Introduction}
  \subsection{Contexte du projet et de la roadmap}
  \subsection{Objectifs de la fonctionnalité choisie}

\section{Analyse préalable}
  \subsection{Enjeux fonctionnels, techniques ou utilisateurs}
  \subsection{Contraintes identifiées}

\section{Conception}
  \subsection{Schéma de base de données (si applicable)}
  \subsection{Diagrammes ou maquettes (si utiles à la compréhension)}
  \subsection{Choix techniques argumentés}

\section{Réalisation}
  \subsection{Développement effectué}
  \subsection{Captures de code ou extraits clés avec explication}
  \subsection{Difficultés rencontrées et gestion}

\section{Tests et validation}
  \subsection{Scénarios de tests mis en place}
  \subsection{Résultats observés}
  \subsection{Outils utilisés (Postman, Lighthouse, etc.)}

\section{Procédure de mise à jour}
  \subsection{Liste des fichiers modifiés}
  \subsection{Étapes nécessaires pour intégrer la fonctionnalité dans le projet existant}
  \subsection{Commandes à exécuter si des dépendances ont été ajoutées}
  \subsection{Précisions sur l’installation ou la mise à jour éventuelle de la base de données}

\section{Bilan personnel}
  \subsection{Apports de cette épreuve}
  \subsection{Retour sur la méthodologie utilisée}
  \subsection{Recommandations éventuelles}

\end{document}
